\documentclass{article}

\usepackage{Sweave}
\begin{document}
\Sconcordance{concordance:lognormDerivation.tex:lognormDerivation.Rnw:%
1 2 1 1 0 7 1 1 2 1 0 3 1 1 2 1 0 1 2 1 0 1 1 6 0 1 1 1 8 7 0 1 1 4 0 1 %
2 3 1}


\title{Derviation of the lognormal distribution}
\maketitle

When simulating a ``genome'' the probability of an exon being captured is drawn from a lognormal distribution. This document shows how the parameters for the distribution were taken from a real pool of 16 samples, captured after the samples were pooled together.

\begin{Schunk}
\begin{Sinput}
> library(data.table)
> library(cnvR)
> library(MASS)
> data(real16) ## Load the real data into the environment
> ## Calculate the median counts (across samples) per exon
> meds <- real16[ , list(medN = median(N)), by = ref][medN > 0]
> ## Fit the median counts (across samples) per exon to lognormal
> fit <- meds[ , fitdistr(medN/sum(medN), "lognormal")]
> signif(fit$estimate, 4)
\end{Sinput}
\begin{Soutput}
 meanlog    sdlog 
-12.3600   0.7393 
\end{Soutput}
\begin{Sinput}
> xvls <- seq(0, 4e-5, length.out = 1000)
> hist(meds[ , medN/sum(medN)],
+      breaks = 1000,
+      xlim = range(xvls),
+      freq = FALSE,
+      border = "darkgrey",
+      col = "lightgrey",
+      xlab = "Probability of capture",
+      main = "")
> lines(xvls, dlnorm(xvls, fit$estimate[1], fit$estimate[2]), lwd = 2)
\end{Sinput}
\end{Schunk}
\includegraphics{lognormDerivation-001}



\end{document}
